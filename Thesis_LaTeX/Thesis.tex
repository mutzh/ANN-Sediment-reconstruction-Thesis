\documentclass[twocolumn]{article}
\usepackage{latex8}

\begin{document}
\title{Reconstructino of Suspended Sediment using Artificial Intelligence}
\author{Dominic Mutzhas}
\maketitle

\begin{abstract} 

\end{abstract} 


\section{Introduction}
As the machine learning(ML) revolution progresses, more and more scientific disciplines are trying to take advantage of it's strenghts in handling complex and non-linear problems. As such it has also found it's way into the field of water management in the form of artificial neural networks (ANN), especially for the prediction of Flow and Sediment.

\section{Background} 
Smartphones are increasingly being used to store personal information as well as to access sensitive data from the Internet and the cloud. Establishment of the identity of a user requesting information from smartphones is a prerequisite for  secure systems in such scenarios. In the past, keystroke-based user identification has been successfully deployed on production-level mobile devices to mitigate the risks associated with naive username/password based authentication. However, these approaches have two major limitations: they are not applicable to services where authentication occurs outside the domain of the mobile
device such as web-based services; and they often overly tax the limited computational capabilities of mobile devices. In this paper, we propose a protocol for keystroke dynamics analysis which allows web-based applications to make use of remote attestation and\cite{nauman2011using} delegated keystroke analysis. The end result is an efficient keystroke-based user identification mechanism that strengthens traditional password protected services
while mitigating the risks of user profiling by \cite{se-cs-collab:nauman10}collaborating malicious web
services. We present a prototype implementation of our protocol using
the popular Android operating system for smartphones.\cite{seo2011user}

\subsection{Some Related Work} 
Establishment of the identity of a user requesting information from smartphones is a prerequisite for  secure systems in such scenarios. In the past, keystroke-based user identification has been successfully deployed on production-level mobile devices to mitigate the risks associated with naive username/password based authentication. However, these approaches have two major limitations: they are not applicable to services where authentication occurs outside the domain of the mobile
device such as web-based services; and they often overly tax the limited computational capabilities of mobile devices. In this paper, we propose a protocol for keystroke dynamics analysis which allows web-based applications to make use of remote attestation and delegated keystroke analysis.

\section{Conclusions}
In the past, keystroke-based user identification has been successfully deployed on production-level mobile devices to mitigate the risks associated with naive username/password based authentication. However, these approaches have two major limitations: they are not applicable to services where authentication occurs outside the domain of the mobile
device such as web-based services.


\input{subfile_0}

% -------------------------------------------------------------------
% add bibliography-related commands here 
\bibliographystyle{IEEEtran}
\bibliography{bibfile_Thesis}   
\end{document}